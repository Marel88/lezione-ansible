% !TeX spellcheck = it_IT
\documentclass{beamer}
\usepackage[english]{babel}
\usepackage[utf8]{inputenc}
\usepackage[T1]{fontenc}

\usetheme{Szeged}
\usecolortheme{beaver}

%\usepackage{libertine}
%\usepackage{opensans}
%\usepackage{quattrocento}

\usepackage{inconsolata}
\usepackage{listings}
\lstset{basicstyle=\ttfamily}

\usepackage{parskip}
\usepackage{xcolor}

\title[Ansible]{Lezione 1: Ansible}
\subtitle{Corso GNU/Linux avanzato e tecnologie aperte primavera 2019}
\author{quel\_tale}
\institute{WEEE Open}
\date[2019]{9 marzo 2019}

\begin{document}
	
\frame{\titlepage}
	
\begin{frame}
\frametitle{Gestione della configurazione}
	Termine generale, applicazioni in vari campi dell'ingegneria.
	
	In ingegneria del software:
	\newline
	
	\begin{quote}
		A discipline applying technical and administrative direction and surveillance to: identify and document the functional and physical characteristics of a configuration item, control changes to those characteristics, record and report change processing and implementation status, and verify compliance with specified requirements.\footnote{ISO/IEC/IEEE 24765:2010 Systems and Software Engineering—Vocabulary, ISO/IEC/IEEE, 2010.}
	\end{quote}
\end{frame}

\begin{frame}
\frametitle{Ma in pratica?}
Controllare, sorvegliare e tracciare le modifiche effettuate a componenti configurabili di un sistema.

Anche \textbf{SVN} e \textbf{git} sono software di gestione della configurazione:
\begin{itemize}
	\item Tracciano le modifiche effettuate sul codice sorgente
	\item Il codice è un componente del sistema (software)
	\item Il codice è configurabile (modificabile)
\end{itemize}
\end{frame}

\begin{frame}
\frametitle{Quindi...}
Ansible è un software di gestione della configurazione:
\begin{itemize}
	\item Controlla le modifiche effettuate ai file o ai database su un server
	\item I file/database sono un componente del sistema (server)
	\item I file/database sono configurabili (file di configurazione, etc...)
\end{itemize}
\end{frame}

\begin{frame}
\frametitle{Un consiglio}

``Versionate'' i file di Ansible con git, non ve ne pentirete!

\vspace{2em}

\begin{alertblock}{Per non reinventare la ruota}
	\href{https://github.com/mattjbarlow/ansible-directory}{github.com/mattjbarlow/ansible-directory}
\end{alertblock}

\begin{alertblock}{Per creare ruoli}
	\texttt{ansible-galaxy init nome-del-modulo}
\end{alertblock}

\begin{alertblock}{Per usare al meglio i ruoli standard}
	\href{https://docs.ansible.com/ansible/latest/}{docs.ansible.com/ansible/latest/}
\end{alertblock}

\end{frame}

\begin{frame}

Fine.

\end{frame}

%\begin{frame}[fragile]
%\begin{lstlisting}[frame=single]
%- asd:
%lel: lol
%lilz: lolz
%\end{lstlisting}

\end{document}