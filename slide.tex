% !TeX spellcheck = it_IT
\documentclass{beamer}
\usepackage[english]{babel}
\usepackage[utf8]{inputenc}
\usepackage[T1]{fontenc}

\usetheme{Szeged}
\usecolortheme{beaver}

%\usepackage{libertine}
%\usepackage{opensans}
%\usepackage{quattrocento}

\usepackage{inconsolata}
\usepackage{listings}
\lstset{basicstyle=\ttfamily}

\usepackage{parskip}
\usepackage{xcolor}

\title[Ansible]{Lezione 1: Ansible}
\subtitle{Corso GNU/Linux avanzato e tecnologie aperte primavera 2019}
\author{quel\_tale}
\institute{WEEE Open}
\date[2019]{9 marzo 2019}

\newcommand{\host}[1]{{\color{blue}{#1}}}
\newcommand{\prompt}[1]{\textbf{#1}}

\begin{document}
	
\frame{\titlepage}

\begin{frame}[fragile]
\frametitle{Metodo da disperati}
\texttt{%
\prompt{tu@\host{tuo-computer}\$} ssh server1.azienda.it\\
\prompt{tu@\host{server1}\$} sudo useradd -Nr -g users fittizio\\
\prompt{tu@\host{server1}\$} sudo passwd fittizio\\
Nuova password: \\
Reimmettere la nuova password: \\
passwd: password aggiornata correttamente\\
\prompt{tu@\host{server1}\$} exit\\
\prompt{tu@\host{tuo-computer}\$} ssh server2.azienda.it\\
\prompt{tu@\host{server2}\$} sudo useradd -Nr -g users fittizio\\
...}
\end{frame}
	
\begin{frame}[fragile]
\frametitle{Metodo classico}
\begin{lstlisting}[basicstyle=\tiny]
#!/bin/bash

for num in $(seq 1 100); do
  ssh server${num}.azienda.it "sudo useradd -Nr -g users fittizio"
  ssh server${num}.azienda.it "echo fittizio:asd | sudo chpasswd"
done

\end{lstlisting}

\vspace{2em}

Se dal fondo dell'aula non si vedesse: è uno script bash con un ``for'' che esegue comandi via ssh sui 100 server, in sequenza.

{\tiny Nota: non ho provato lo script e non intendo provarlo}
\end{frame}

\begin{frame}[fragile]
\frametitle{Metodo che vedremo oggi}
\begin{lstlisting}
- name: Ensure user 'fittizio'
  user:
    name: fittizio
    group: users
    comment: System user for Fittizio software
    system: true
    password: $6$TkGclvyVl847NeuO$F9fsQ/VbMj29l4egyFKPB.lSpFvWdmfk2Zlp9U0xHoP1ipTu9gjoXMRgQkIqb.aIERVR3ecbsPsABuftYIlTY/
    
\end{lstlisting}

{\small La password andava cifrata. Quello è l'hash.}
\end{frame}

\begin{frame}
\frametitle{Gestione della configurazione}
	Termine generale, applicazioni in vari campi dell'ingegneria.
	
	In ingegneria del software:
	\newline
	
	\begin{quote}
		A discipline applying technical and administrative direction and surveillance to: identify and document the functional and physical characteristics of a configuration item, control changes to those characteristics, record and report change processing and implementation status, and verify compliance with specified requirements.\footnote{ISO/IEC/IEEE 24765:2010 Systems and Software Engineering—Vocabulary, ISO/IEC/IEEE, 2010.}
	\end{quote}
\end{frame}

\begin{frame}
\frametitle{Ma in pratica?}
Controllare, sorvegliare e tracciare le modifiche effettuate a componenti configurabili di un sistema.

Anche \textbf{SVN} e \textbf{git} sono software di gestione della configurazione:
\begin{itemize}
	\item Tracciano le modifiche effettuate sul codice sorgente
	\item Il codice è un componente del sistema (software)
	\item Il codice è configurabile (modificabile)
\end{itemize}
\end{frame}

\begin{frame}
\frametitle{Quindi...}
Ansible è un software di gestione della configurazione:
\begin{itemize}
	\item Controlla le modifiche effettuate ai file o ai database su un server
	\item I file/database sono un componente del sistema (server)
	\item I file/database sono configurabili (file di configurazione, etc...)
\end{itemize}
\end{frame}

\begin{frame}
\frametitle{Breve storia}
\textbf{Prima generazione: CFEngine (1993), Bcfg2 (2004)}

Mondo accademico, centralità di aspetti teorici, automazione e astrazione

\textbf{Seconda generazione: Puppet (2005), Chef (2009)}

Aziende, semplificare la vita ai sistemisti, deployment su larga scala

\textbf{Terza generazione: Ansible (2009), Salt (2011)}

Semplicità di sviluppo, velocità, automazione anche parziale ``basta che si faccia''

\end{frame}

\begin{frame}
\frametitle{Perché Ansible?}

\textbf{Facile da iniziare a usare}: agentless, playbook ``lineari'', automazione anche parziale della configurazione, sintassi in formati standard e noti (YAML, Jinja2)

\textbf{Largamente supportato}: può configurare praticamente qualsiasi distribuzione GNU/Linux, BSD, Windows, OS X, router professionali, container (Docker), etc...

\textbf{Comunità attiva}: molti ruoli disponibili su Ansible Galaxy, manuale dettagliato e con molti esempi

Tuttavia, alcuni svantaggi: velocità limitata, non perfettamente idempotente e dichiarativo, ``drift'' della configurazione (modello push)

\end{frame}

\begin{frame}
\frametitle{Consigli finali}

Leggete il manuale e cercate su Internet, prima di reinventare la ruota.

``Versionate'' i playbook e i ruoli con git.

\vspace{1em}

\begin{alertblock}{Un buon punto di partenza}
	\href{https://github.com/mattjbarlow/ansible-directory}{github.com/mattjbarlow/ansible-directory}
\end{alertblock}

\begin{alertblock}{Per creare ruoli}
	\texttt{ansible-galaxy init nome-del-modulo}
\end{alertblock}

\begin{alertblock}{Per usare al meglio i ruoli standard}
	\href{https://docs.ansible.com/ansible/latest/}{docs.ansible.com/ansible/latest/}
\end{alertblock}

\end{frame}

\begin{frame}

{\Huge Grazie per l'attenzione}

\end{frame}

\begin{frame}

{\Huge Domande?}

\vspace{2em}

Oppure andate in pace, sono le 20.30

\end{frame}


\end{document}